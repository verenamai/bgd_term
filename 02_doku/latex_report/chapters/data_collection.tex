\chapter{Software Evaluation}
\label{chap:data_collection}

In dieser Semesterarbeit wurde keine ausführliche Evaluation vorgenommen, da dies nicht Teil der Zielsetzung war. Es wurden sich rudimentär mögliche Software angeschaut, die Vor- und Nachteile abgewogen und anschliessend eine Auswahl getroffen. Im folgenden sind die Software, Vor- und Nachteile und Entscheidungen dokumentiert.

\section{Container}
\label{sec:container}

Ein Container ist eine Gef{\"a}ss, welches es erm{\"o}glicht Anwendungen auf unterschiedlichen Plattformen ohne grossen Aufwand zu verteilen. Containerisierte Software, die sowohl f{\"u}r Linux- als auch f{\"u}r Windows-basierte Anwendungen verf{\"u}gbar ist, l{\"a}uft immer gleich, unabh{\"a}ngig von der Infrastruktur. Das Container-Image beinhaltet alles enth{\"a}lt was zum Ausf{\"u}hren einer Anwendung erforderlich ist: Code, Laufzeit, Systemtools, Systembibliotheken und Einstellungen.

Container-Images werden zur Laufzeit zu Containern und im Falle von Docker-Containern - Images werden zu Containern, wenn sie auf der Docker Engine laufen. Container isolieren Software von ihrer Umgebung und stellen sicher, dass sie trotz Unterschieden, z.B. zwischen Entwicklung und Staging, einheitlich funktioniert.

\textbf{Entscheidung}
Für den ersten Durchstich, welcher lokal auf dem Notebook laufen soll wird Docker mit docker 


\subsection{Docker}
\label{sec:Docker}

\subsection{Kubernetes}
\label{sec:Kubernetes}


\section{Streaming-Software}
\label{sec:streaming}

Streaming-Software wurde entwickelt, um den Datenfluss zwischen unterschiedlichen Softwaresystemen und Datenbanken und Datenverarbeitung zu automatisieren. \footnote{\label{foot:1} https://freshcodeit.com/freshcode-post/top-5-enterprise-etl-tools}. 

\subsection{Apache NiFi}

Apache NiFi ist ein Open-Source-Projekt, entwickelt von der Apache Software Foundation. Es basiert auf dem Konzept der Datenflussprogrammierung, was bedeutet, dass dieses ETL-Tool es erm{\"o}glicht, die Piplienes visuell zusammenzustellen und fast ohne Programmierung auszuf{\"u}hren. NiFi kann mit verschieden Quellen arbeiten: Zum Beispiel RabbitMQ, JDBC-Abfrage, Hadoop, MQTT, UDP-Socket, etc. Um die Daten zu prozessieren bietet das Tools M{\"o}glichkeiten um die Daten zu filtern, anzupassen, zu verbinden, aufzuteilen, zu erweitern und zu {\"u}berpr{\"u}fen. 

Vorteile: 
\begin{itemize}
  \item Perfekte Umsetzung des Konzepts der Datenflussprogrammierung
  \item M{\"o}glichkeit, bin{\"a}re Daten zu verarbeiten
  \item Datenherkunft
\end{itemize}

Nachteile: 
\begin{itemize}
  \item Vereinfachte Benutzeroberfl{\"a}che 
  \item Fehlende Live-{\"U}berwachung und Statistiken pro Datensatz
\end{itemize}

\subsection{Streamsets}

todo: hier noch kurz etwas allgemeines zu Streamsets schreiben.

Vorteile: 
\begin{itemize}
  \item Jeder Prozessor hat individuelle Statistiken pro Datensatz mit sch{\"o}ner Visualisierung f{\"u}r effektives Debugging. 
  \item Attraktive Benutzeroberfl{\"a}che
  \item Gutes ETL-Tool f{\"u}r Streaming oder dateibasierte Daten
\end{itemize}

Nachteile: 
\begin{itemize}
  \item Fehlende wiederverwendbare JDBC-Konfiguration 
  \item Die {\"A}nderung einer Einstellung eines Prozessors erfordert das Stoppen des gesamten Datenflusses.
\end{itemize}
 
Alle Daten, die Sie in Streamsets eingeben, werden automatisch in austauschbare Datens{\"a}tze umgewandelt. Das g{\"a}ngige Format ist f{\"u}r ein reibungsloses Streaming ausgelegt. Im Gegensatz zu Apache Nifi zeigt dieses ETL-Tool keine Warteschlangen zwischen Prozessoren an. Wenn Sie verschiedene Formate verwenden m{\"o}chten, muss Apache Nifi von einer Version des Prozessors zur anderen wechseln. Streamsets erm{\"o}glichen es, diese Manipulationen zu vermeiden. Anstatt nur einen Prozessor zu stoppen, m{\"u}ssen Sie den gesamten Datenfluss stoppen, um die Einstellungen zu {\"a}ndern. 

\subsection{Airflow}

Airflow ist eine Plattform f{\"u}r die Gestaltung, Erstellung und Verfolgung von Workflows. Sie kann mit Cloud Services wie GCP, Azure und AWS verwendet werden und es besteht die M{\"o}glichkeit, Aiflow auf Kubernetes mit Astronomer Enterprise zu betreiben. Die Workflows sind in Python geschrieben, aber die Schritte selbst k{\"o}nnen beliebig ausgef{\"u}hrt werden. Aiflow wurde als perfekter flexibler Aufgabenplaner entwickelt. 

Vorteile: 
\begin{itemize}
  \item Geeignet f{\"u}r verschiedene Arten von Aufgaben
  \item Benutzerfreundliche Oberfl{\"a}che f{\"u}r eine {\"u}bersichtliche Visualisierung
  \item Skalierbare L{\"o}sung
\end{itemize}

Nachteile: 
\begin{itemize}
  \item Ist nicht geeignet f{\"u}r Streaming-Auftr{\"a}ge 
  \item Erfordert zus{\"a}tzliche Operatoren
\end{itemize}

F{\"u}r die Semesterarbeit wird Streamsets eingesetzt. Airflow wurde ausgeschlossen, weil es sich f{\"u}r Streaming nicht eigent. 

\section{Messaging: Kafka}
\label{sec:messaging}

Apache Kafka ist eine Open-Source-Plattform für Streaming, die es ermöglicht, Daten mit hohem Durchsatz und geringer Latenzzeit zu übertragen. Kafka wurde ursprünglich bei LinkedIn konzipiert und 2011 Open-Source entwickelt und hat seitdem eine breite Akzeptanz in der Community gefunden. Mittlerweile ist es das bevorzugte Echtzeit-Messaging-System in der Branche. Ein sogenannter Kafka Producer schreibt (ver{\"o}ffentlichen) Nachrichten und ein Kafka Consumer liesst (abonnieren) diese. Weitere Informationen zu Kafka und die Funktionen sind der Herstellerseite zu entnehmen. \footnote{\label{foot:2} https://kafka.apache.org/intro}


\section{Datenbanken}
\label{sec:datenbanken}


Die Entscheidung für MongoDB und Elasticsearch wurden ganz am Anfang der Semesterarbeit gefällt. Für den 

da zum einen MongoDB als reiner Datenspeicher für später sehr einfach aufzusetzen ist und zum anderen alles an den gesendeten Twitter.json und immer wieder unterschiedlichen 

\subsection{MongoDB}

\subsection{Elasticsearch}


\section{Dashboard}
\label{sec:dashboard}
\subsection{Kibana}


\subsection{Entscheidung}

Container: Docker
Streaming-Software: Streamsets
IM 
